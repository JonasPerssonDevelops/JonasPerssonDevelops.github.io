\documentclass{article}
\usepackage[margin=1.5in]{geometry}
\usepackage{graphicx}
\usepackage[section]{placeins}
\usepackage{hyperref}

\begin{document}
	
	\title { Portland State CS 410 Summer 2022 - Intro to Web Development Journal }
	\author {Jonas Persson}
	\maketitle
	
	\newpage

\section*{template}
\begin{itemize}
	\item definition:
	\item description:
\end{itemize}

\section*{7/31/2022}
\begin{itemize}
	\item Created the GitHub repository \newline \href{https://github.com/JonasPerssonDevelops/JonasPerssonDevelops.github.io}{https://github.com/JonasPerssonDevelops/JonasPerssonDevelops.github.io} \newline 
	for containing the source code and deploying it to \newline \href{https://jonasperssondevelops.github.com}{https://jonasperssondevelops.github.io}
	\item Studied the requirements in \texttt{Final Project - Summer.pdf}
	\item Wrote up a top-design design for the web app, deciding to go with multiple HTML files that are populated with elements and content dynamically through JavaScript, especially the footer and navbar
	\item Researched:
		\begin{itemize} 
        	\item Whether to use HTML entities or not for content characters (answer: only for $<$, $>$, and \&)
        	\item Storing data in the HTML file to identify the page in order to run custom functions for each page in JavaScript (which can be accomplished using data-* attributes in the HTML)
        	\item JavaScript file, directory, and identifier naming conventions
        	\item Directory structure for assets for web apps
		\end{itemize}
		\item Tested with a skeleton HTML file and a .js file whether JavaScript can identify a page by data-* attributes. Success! Happy days are here again...

\end{itemize}

\section*{8/1/2022}
\begin{itemize}
	\item Continued to study the requirements in \texttt{Final Project - Summer.pdf} and top down plan the web app
	\item Researched semantic HTML for the project cards, and for the overall structure of each HTML page, deciding to go with a article for each card and a header, main, and footer on each page
	\item Decided that the footer should have social media and github repo links

	\item Wrote a first version of the GitHub repository README.md and committed it
	\item Checked out some professional bio websites for content inspiration \href{https://www.chrisducker.com/}{https://www.chrisducker.com/}, \href{https://www.marieforleo.com/}{https://www.marieforleo.com/}
	\item Researched:
		\begin{itemize} 
			\item GitHub flavor markdown syntax
			\item How to use tags header, main, article, section, aside, and footer together on the same page
			\item Placing the nav outside or inside the header (either seems to be ok)
			\item Website trends and how standard layouts for 2021 websites look
			\item How to nest main, section, and article content
			\item Spent 1 hour looking at different websites with this guide \newline \href{https://www.intechnic.com/blog/top-50-examples-best-responsive-designs/}{https://www.intechnic.com/blog/top-50-examples-best-responsive-designs/}
			\href iphone viewport sizes
		\end{itemize}
	\item Wrote up a high-level architecture for each page with the following structure: \texttt{<nav>, <header>, <main>, sections, articles...., <footer>}
\end{itemize}


\section*{8/2/2022}
\begin{itemize}
	\item Researched:
		\begin{itemize} 
	 			\item Creating entirely dynamic pages versus several pages with dynamic elements; Going with several pages to allow the user to bookmark different pages easier
	 			\item The directory structure of GitHub Pages
	 			\item Using CSS Grid instead of bootstrap to have more control over styling
		\end{itemize}
	\item Reimplemented the navbar from the homeworks
	\item Added ARIA accessibility markup to the navigation bar

\end{itemize}


\section*{8/3/2022}
\begin{itemize}
	\item Implemented a responsive header and navbar with CSS grid to display decorative text next to a responsive navbar.
	\item Researched:
		\begin{itemize} 
	 			\item How to keep my name as decorative text hidden from screen readers so it isn't read out on every page. Solved with aria-hidden="true"
		\end{itemize}
	\item Added ARIA accessibility markup to the navigation bar.
	\item Linked a Google font to style the decorative text in

\end{itemize}

\section*{8/4/2022}
\begin{itemize}
	\item Implemented JavaScript functions that dynamically build the navigation bar. 
	\item Hid the social button icons from screen readers with \texttt{aria-hidden = "true"} as they don't add any functionality for people who are blind and they aren't part of the content of this personal website.
	\item Hid the social button icon links from screen readers with \texttt{tabindex = -1}
	\item Researched:
		\begin{itemize} 
	 			\item How to add link text in JavaScript.
	 			\item How to add a node in JavaScript.
	 			\item Social buttons and rules for using copyrighted images from Meta Platforms and GitHub with them
	 			\item Is the syntax double or single quotes for JavaScript code
	 			\item How to center cell content in CSS grid
	 			\item Method for getting elements by id in JavaScript
	 			\item What the proper method for adding text as a child to an HTML tag is
	 			\item Correct semantic HTML for social button links
	 			\item Hiding redundant links from screen readers with \texttt{tabindex = -1}
		\end{itemize} 
\end{itemize}

\section*{8/5/2022}
\begin{itemize}
	\item Added, tested, and tweaked responsive font sizing with \texttt{rem} and a formula for font size: \texttt{font-size: calc(90\% + 0.5vw);} in order to achieve about 55 characters / 9 words per line on mobile
	\item Added, tested, and tweaked responsive image sizing with \texttt{rem} and \texttt{@media} queries. 
	\item Tried a new CSS grid format for the footer but the complexity didn't add any benefit so abandon ship, and reverted to a flex-box space-evenly solution instead.
	\item Formatted the markdown file as text wasn't centering properly.
	\item Fixed the social button links so they open in a new tab.
	\item Tried to get the footer to stick to the bottom but gave up and saw that major websites don't bother to do that.
	\item Tested different color schemes.
	\item Tested resizing images proportionally to the viewport width but ran into problems as svg images don't have sizes so it appears impossible to scale them relatively, couldn't find a solution anyways
	\item Researched:
		\begin{itemize} 
	 			\item Semantic footer examples
	 			\item Optimal number of characters per line: 50-75 according to \href{https://baymard.com/blog/line-length-readability}{https://baymard.com/blog/line-length-readability}
	 			\item Project directory structure for storing documentation
		\end{itemize}
\end{itemize}



\section*{8/6/2022}
\begin{itemize}
	\item Styled the navigation buttons to make them look like contemporary CSS web design, but also look professional/conservative in styling (because it's a professional bio / personal website).
	\item Changed the site links button hover behavior 
	\item Tweaked the header and footer gradients.
	\item Gave main a circular look with rounded edges using a transition div
	\item Added padding to main so that the content of main doesn't touch the sides of the windows.
	\item Restyled the links in the footer using a CSS descendant selector removing the default colors and underline
	\item Added major colors for gradients and titles on different pages as variable names in
	 \texttt{:root} 
	\item Changed title font to Oswald
	\item Restyled the \texttt{media} queries to position the text better for mobile and desktop
	\item Researched:
		\begin{itemize} 
	 			\item Semantic footer examples
	 			\item How to combine calc and Viewport width / vw to have responsive text sizing and padding for main
		\end{itemize}
\end{itemize}


\section*{8/7/2022}
\begin{itemize}
	\item Added a profile photo and designed another responsive CSS grid for content in main on \texttt{index.html}
	\item Designed CSS text bubble divs for displaying quotes
	\item Added a page for work and other experience
	\item Added some favorite quotes to \texttt{index.html}
	\item Identified and fixed a \texttt{@media} bug with CSS grid and the \texttt{fr} units that was breaking the responsive design at some resolutions
	\item Researched:
		\begin{itemize} 
	 			\item How to include quotes using semantic HTML
	 			\item Centering elements with the help of a container and flex box
	 			\item Semantic bold text
		\end{itemize}
\end{itemize}




\section*{template}
	
\begin{itemize}
	\item definition:
	\item description:
\end{itemize}


















































	
\end{document}